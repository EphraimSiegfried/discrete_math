
%% General definitions
\documentclass{article} %% Determines the general format.
\usepackage{a4wide} %% paper size: A4.
\usepackage[utf8]{inputenc} %% This file is written in UTF-8.
%% Some editors on Windows cannot save files in UTF-8.
%% If there is a problem with special characters not showing up
%% correctly, try switching "utf8" to "latin1" (ISO 8859-1).
\usepackage[T1]{fontenc} %% Format of the resulting PDF file.
\usepackage{fancyhdr} %% Package to create a header on each page.
\usepackage{lastpage} %% Used for "Page X of Y" in the header.
%% For this to work, you have to call pdflatex twice.
\usepackage{enumerate} %% Used to change the style of enumerations (see below).

\usepackage{amssymb} %% Definitions for math symbols.
\usepackage{amsmath} %% Definitions for math symbols.
\usepackage{amsthm} % Definiton for Proofs
\usepackage{pifont}
\usepackage{tikz}  %% Pagacke to create graphics (graphs, automata, etc.)
\usetikzlibrary{automata} %% Tikz library to draw automata
\usetikzlibrary{arrows}   %% Tikz library for nicer arrow heads


%% Left side of header
\lhead{\course\\\semester\\Exercise \homeworkNumber}
%% Right side of header
\rhead{\authorname\\Page \thepage\ of \pageref{LastPage}}
%% Height of header
\usepackage[headheight=36pt]{geometry}
%% Page style that uses the header
\pagestyle{fancy}

\newcommand{\authorname}{Alex Lutsch\\Ephraim Siegfried }
\newcommand{\semester}{Fall Semester 2023}
\newcommand{\course}{Discrete Mathematics in Computer Science}
\newcommand{\homeworkNumber}{4}


\begin{document}


\section*{Exercise \homeworkNumber.1}
\begin{enumerate}[(a)]
	\item We have that \( \left\{ 2,3 \right\} \times \emptyset
	      = \left\{ \left< a,b \right> \mid a \in \left< 2, 3\right>
	      \text{ and } b \in \emptyset \right\}\). But there exists no
	      \( b \) for which \( b \in \emptyset \) is true.
	      Therefore the statement is wrong and the correct statement would be
	      \( \left\{ 2,3 \right\} \times \emptyset = \emptyset \).
	\item With the definition of the cartesian product we have that
	      \( \left\{ \left< 1,0 \right> \right\} \times \left\{ 0 \right\}
	      = \left\{ \left< \left< 1,0 \right>, 0 \right>\right\}  \) = S.
	      Neither of the elements in the right-hand side of the statement are elements of \( S \),
	      therefore the statement is wrong.
	\item The order matters in tuples, therefore \( \left< 1,2 \right> \neq \left< 2,1 \right> \).
	      Thus the set on the left-hand side is not a subset of the set in the right-hand side and vice versa,
	      which means the two sets are not equal. Therefore the statement is wrong.
	\item The cartesian product of the two sets
	      \( C = \left\{ 0, 1, 2 \right\} \times \left\{ 3,4,5 \right\}  \) results in a set of tuples.
	      The set \( \left\{ 2,4 \right\}  \)  is not a tuple.
	      Therefore it cannot be element of \( C \), which
	      means that the statement is wrong.

\end{enumerate}

\section*{Exercise \homeworkNumber.2}
\begin{enumerate}[(a)]
	\item $A = \{2,3,5\} = B$
	\item $\lvert A \cup B\lvert = 4$  so $A \cup B = \{1,6,4,x\}$ \\
	      $\lvert A \times B \lvert = 6$ so $\lvert A\lvert  * \lvert B\lvert  = 6$ \\
	      $\langle\langle 1,6 \rangle,4\rangle \in (A \times A) \times B$ so $ \{1,6\} \in A$ and $ \{4\} \in B$ \\
	      $ A = \{1,6,3\}$ $B = \{3,4\}$
\end{enumerate}
\section*{Exercise \homeworkNumber.3}
\begin{enumerate}[(a)]
	\item \( R_{1} = \left\{ \left< a,a \right>, \left< c,b \right>,\left< b,c \right> \right\}  \)
	\item This is not possible, which I will show by contradiction.
	      Suppose it was possible, then per definition \( \left< a,b \right> \in  R_{2}\).
	      Because the relation should also be symmetric, we know that \( \left< b,a \right> \in R_{2}\).
	      With the transitivity property, it must be the case that if \( \left< a,b \right> \in R_{2} \)
	      and \( \left< b,a \right> \in R_{2}\), then \( \left< a,a \right> \in R_{2} \).
	      But if \( \left< a,a \right> \in R_{2} \) then the relation is not irreflexive.
	      This is a contradiction and therefore this relation cannot exist.

	      % \item \( R_{2} = \left\{ \left< a,b \right>, \left< b,a \right>,  \right\}  \)
\end{enumerate}

\section*{Exercise \homeworkNumber.4}
We examine the following binary Relation: $R = \{ \langle i,j*i \rangle \lvert i,j \in \mathbb N_0\}$ \\
Since $ i,j \in \mathbb N_0 , i = (j*i)$ because every number of $i$ can be formed with setting $j$ one or zero and the multiplication of two natural numbers is a natural number which is $i$.\\
It is reflexive since $i = (j*i)$. \\
It's not irreflexive since its reflexive. \\
It's symetric since any natural number appears on both sides of the tuple. \\
It's not asymmetric or antisymmetric since its symetric. \\
Since both elements of the tuple represent all possible natural numbers we can guarantee transitivity. \\
\end{document}
