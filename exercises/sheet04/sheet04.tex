
%% General definitions
\documentclass{article} %% Determines the general format.
\usepackage{a4wide} %% paper size: A4.
\usepackage[utf8]{inputenc} %% This file is written in UTF-8.
%% Some editors on Windows cannot save files in UTF-8.
%% If there is a problem with special characters not showing up
%% correctly, try switching "utf8" to "latin1" (ISO 8859-1).
\usepackage[T1]{fontenc} %% Format of the resulting PDF file.
\usepackage{fancyhdr} %% Package to create a header on each page.
\usepackage{lastpage} %% Used for "Page X of Y" in the header.
%% For this to work, you have to call pdflatex twice.
\usepackage{enumerate} %% Used to change the style of enumerations (see below).

\usepackage{amssymb} %% Definitions for math symbols.
\usepackage{amsmath} %% Definitions for math symbols.
\usepackage{amsthm} % Definiton for Proofs
\usepackage{pifont}
\usepackage{tikz}  %% Pagacke to create graphics (graphs, automata, etc.)
\usetikzlibrary{automata} %% Tikz library to draw automata
\usetikzlibrary{arrows}   %% Tikz library for nicer arrow heads


%% Left side of header
\lhead{\course\\\semester\\Exercise \homeworkNumber}
%% Right side of header
\rhead{\authorname\\Page \thepage\ of \pageref{LastPage}}
%% Height of header
\usepackage[headheight=36pt]{geometry}
%% Page style that uses the header
\pagestyle{fancy}

\newcommand{\authorname}{Alex Lutsch\\Ephraim Siegfried }
\newcommand{\semester}{Fall Semester 2023}
\newcommand{\course}{Discrete Mathematics in Computer Science}
\newcommand{\homeworkNumber}{4}


\begin{document}


\section*{Exercise \homeworkNumber.1}

\section*{Exercise \homeworkNumber.2}
\begin{enumerate}[(a)]
    \item $A = \{2,3,5\} = B$
    \item $\lvert A \cup B\lvert = 4$  so $A \cup B = \{1,6,4,x\}$ \\
        $\lvert A \times B \lvert = 6$ so $\lvert A\lvert  * \lvert B\lvert  = 6$ \\
        $\langle\langle 1,6 \rangle,4\rangle \in (A \times A) \times B$ so $ \{1,6\} \in A$ and $ \{4\} \in B$ \\
        $ A = \{1,6,3\}$ $B = \{3,4\}$
\end{enumerate}
\section*{Exercise \homeworkNumber.3}

\section*{Exercise \homeworkNumber.4}
We examine the following binary Relation: $R = \{ \langle i,j*i \rangle \lvert i,j \in \mathbb N_0\}$ \\
Since $ i,j \in \mathbb N_0 , i = (j*i)$ because every number of $i$ can be formed with setting $j$ one or zero and the multiplication of two natural numbers is a natural number which is $i$.\\
It is reflexive since $i = (j*i)$. \\
It's not irreflexive since its reflexive. \\
It's symetric since any natural number appears on both sides of the tuple. \\
It's not asymmetric or antisymmetric since its symetric. \\
Since both elements of the tuple represent all possible natural numbers we can guarantee transitivity. \\
\end{document}
