
%% General definitions
\documentclass{article} %% Determines the general format.
\usepackage{a4wide} %% paper size: A4.
\usepackage[utf8]{inputenc} %% This file is written in UTF-8.
%% Some editors on Windows cannot save files in UTF-8.
%% If there is a problem with special characters not showing up
%% correctly, try switching "utf8" to "latin1" (ISO 8859-1).
\usepackage[T1]{fontenc} %% Format of the resulting PDF file.
\usepackage{fancyhdr} %% Package to create a header on each page.
\usepackage{lastpage} %% Used for "Page X of Y" in the header.
%% For this to work, you have to call pdflatex twice.
\usepackage{enumerate} %% Used to change the style of enumerations (see below).

\usepackage{amssymb} %% Definitions for math symbols.
\usepackage{amsmath} %% Definitions for math symbols.
\usepackage{amsthm} % Definiton for Proofs

\usepackage{tikz}  %% Pagacke to create graphics (graphs, automata, etc.)
\usetikzlibrary{automata} %% Tikz library to draw automata
\usetikzlibrary{arrows}   %% Tikz library for nicer arrow heads


%% Left side of header
\lhead{\course\\\semester\\Exercise \homeworkNumber}
%% Right side of header
\rhead{\authorname\\Page \thepage\ of \pageref{LastPage}}
%% Height of header
\usepackage[headheight=36pt]{geometry}
%% Page style that uses the header
\pagestyle{fancy}

\newcommand{\authorname}{Alex Lutsch\\Kemal Yönet\\Ephraim Siegfried }
\newcommand{\semester}{Fall Semester 2023}
\newcommand{\course}{Discrete Mathematics in Computer Science}
\newcommand{\homeworkNumber}{3}


\begin{document}



\section*{Exercise \homeworkNumber.1}
$A,B \subseteq \{1,...,10\}$ \\
$\lvert B \lvert = 3$, so we choose arbitrarily $B = \{1,2,3\}$. \\
$\lvert \mathcal P(A) \lvert = 16  \iff \lvert A \lvert = \log_2(16) = 4$ \\
$\lvert A \cup B \lvert = 5$ so to reach sum of 5, A has to share 2 Elements with B while having 4 in total.\\
$A = \{2,3,4,5\}$ satisfies these conditions.




\section*{Exercise \homeworkNumber.2}

\begin{enumerate}[(a)]
	\item $\{e1,e4,e6,e7\} = 01001011$
	\item $01110110 = \{e1,e2,e3,e5,e6\}$
	\item
		$A \cup B \iff $ A bitwise OR B\\
		$A \cap B \iff $ A bitwise AND B\\
		$ \lnot A \iff $ bitwise NOT A\\


\end{enumerate}



\section*{Exercise \homeworkNumber.3}

\begin{enumerate}[(a)]
	\item
	\item
\end{enumerate}



\section*{Exercise \homeworkNumber.4}

\begin{enumerate}[(a)]
	\item
	\item
\end{enumerate}

\section*{Exercise \homeworkNumber.5}
\begin{proof}
We want to show that $\mathbb Q$ is countable. We already know that $\mathbb Q_+$ is countable. \\
We want to form a bijection from $\mathbb Q_+$ to $\mathbb Q_-$ to show that $\lvert \mathbb Q_+ \lvert = \lvert \mathbb Q_- \lvert$ and hence $\mathbb Q_-$ is countable too.
We define $f:\mathbb Q_+ \rightarrow \mathbb Q_-$ as $f(x) = -x$. \\
We can define $\mathbb Q = \mathbb Q_+ \cup \mathbb Q_-$, and we know that the union of two countable sets is countable. \\
\end{proof}


\end{document}
