
%% General definitions
\documentclass{article} %% Determines the general format.
\usepackage{a4wide} %% paper size: A4.
\usepackage[utf8]{inputenc} %% This file is written in UTF-8.
%% Some editors on Windows cannot save files in UTF-8.
%% If there is a problem with special characters not showing up
%% correctly, try switching "utf8" to "latin1" (ISO 8859-1).
\usepackage[T1]{fontenc} %% Format of the resulting PDF file.
\usepackage{fancyhdr} %% Package to create a header on each page.
\usepackage{lastpage} %% Used for "Page X of Y" in the header.
%% For this to work, you have to call pdflatex twice.
\usepackage{enumerate} %% Used to change the style of enumerations (see below).

\usepackage{amssymb} %% Definitions for math symbols.
\usepackage{amsmath} %% Definitions for math symbols.
\usepackage{pifont}
\usepackage{tikz}  %% Pagacke to create graphics (graphs, automata, etc.)
\usetikzlibrary{automata} %% Tikz library to draw automata
\usetikzlibrary{arrows}   %% Tikz library for nicer arrow heads


%% Left side of header
\lhead{\course\\\semester\\Exercise \homeworkNumber}
%% Right side of header
\rhead{\authorname\\Page \thepage\ of \pageref{LastPage}}
%% Height of header
\usepackage[headheight=36pt]{geometry}
%% Page style that uses the header
\pagestyle{fancy}

\newcommand{\authorname}{Alex Lutsch\\Ephraim Siegfried }
\newcommand{\semester}{Fall Semester 2023}
\newcommand{\course}{Discrete Mathematics in Computer Science}
\newcommand{\homeworkNumber}{3}
\newcommand{\plane}{\text{\ding{40}}}
\newcommand{\heart}{\text{\ding{95}}}
\newcommand{\flower}{\text{\ding{170}}}

\begin{document}



\section*{Exercise \homeworkNumber.1}

\begin{enumerate}[(a)]
	\item
	\item
\end{enumerate}



\section*{Exercise \homeworkNumber.2}

\begin{enumerate}[(a)]
	\item
	\item
\end{enumerate}



\section*{Exercise \homeworkNumber.3}
We will refute the statement for all sets \( A,B \) it holds that \( |A| < |A \cup B| \). Let \( A = B \), then we have \( A = A \cup B \). In this case it holds that \( |A| = |A \cup B| \), which contradicts the statement.



\section*{Exercise \homeworkNumber.4}

Let \( g \colon \left\{ n \in \mathbb{N} \mid n \text{ mod } 2 = 0 \\ \right\} \to \mathbb{Z} \colon g(n) = -\frac{n}{2} \).
The function \( g(x) \) is a bijective function because it maps every
even natural number to every negated natural number (which is injective and surjective). \\
Let \( s \colon \left\{ n \in \mathbb{N} \mid n \text{ mod } 2 \neq  0 \\ \right\} \to \mathbb{Z} \colon g(n) = \frac{n+1}{2} \).
The function \( s(x) \) is a bijective function because it maps every odd natural number to a natural number
(it is a subset of the natural numbers). \\
The union of two bijective functions is bijective, therefore
\begin{equation*}
	f(n) =
	\begin{cases}
		-\frac{n}{2},  & \text{if } n \text{ is even}, \\
		\frac{n+1}{2}, & \text{otherwise},
	\end{cases}
	\quad \text{for } n \in \mathbb{N},
\end{equation*}
is a bijective function from \( \mathbb{N} \) to \( \mathbb{Z} \) and thus \( \mathbb{Z} \) is countable.


\section*{Exercise \homeworkNumber.5}


\section*{Exercise \homeworkNumber.6}
I will first proof that the set \( S \) containing all finite strings that consist
only of symbols from the set \(  \Sigma = \left\{ \plane, \heart, \flower \right\}  \) is countable. \\
For each \( n \in \mathbb{N} \), there are \( 3^n \) strings of length \( n \), which is finite.
We can then enumerate all strings by length, starting with strings of length 0, then length 1, and so on.
Within each length \( n \), we enumerate the \( 3^n \) strings in some specific order.
By mapping each string to a unique natural number based on its position in the complete enumeration, starting from 0, we establish a one-to-one correspondence between \( S \) and \( \mathbb{N} \).
Thus, the set \( S \) is countable. \\
Since all strings in the set of the tarradiddles \( T \) consist of symbols from \( \Sigma \) we have that \( T \subseteq S \). Because the subset of a countable set is countable, \( T \) is countable.



\end{document}
