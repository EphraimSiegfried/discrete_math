
%% General definitions
\documentclass{article} %% Determines the general format.
\usepackage{a4wide} %% paper size: A4.
\usepackage[utf8]{inputenc} %% This file is written in UTF-8.
%% Some editors on Windows cannot save files in UTF-8.
%% If there is a problem with special characters not showing up
%% correctly, try switching "utf8" to "latin1" (ISO 8859-1).
\usepackage[T1]{fontenc} %% Format of the resulting PDF file.
\usepackage{fancyhdr} %% Package to create a header on each page.
\usepackage{lastpage} %% Used for "Page X of Y" in the header.
%% For this to work, you have to call pdflatex twice.
\usepackage{enumerate} %% Used to change the style of enumerations (see below).

\usepackage{amssymb} %% Definitions for math symbols.
\usepackage{amsmath} %% Definitions for math symbols.

\usepackage{tikz}  %% Pagacke to create graphics (graphs, automata, etc.)
\usetikzlibrary{automata} %% Tikz library to draw automata
\usetikzlibrary{arrows}   %% Tikz library for nicer arrow heads

%% Left side of header
\lhead{\course\\\semester\\Exercise \homeworkNumber}
%% Right side of header
\rhead{\authorname\\Page \thepage\ of \pageref{LastPage}}
%% Height of header
\usepackage[headheight=36pt]{geometry}
%% Page style that uses the header
\pagestyle{fancy}

\newcommand{\authorname}{Alex Lutsch\\Ephraim Siegfried }
\newcommand{\semester}{Fall Semester 2023}
\newcommand{\course}{Discrete Mathematics in Computer Science}
\newcommand{\homeworkNumber}{9}


\begin{document}

\section*{Exercise \homeworkNumber.1}
\begin{enumerate}
	\item This statement is false. This will be shown with a counterexample.Let \( G =  (\left\{ a,b,c \right\},\left\{ \left\{ a,b \right\}, \left\{ b,c \right\}  \right\}    \). G is connected, because we can reach every vertex from every vertex. The Graph \( G' = (\left\{ a,c \right\}, \emptyset) \) is an induced subgraph of \( G \), since it is formed from a subset of the vertices from \( G \) and the vertices \( a,c \) are not connected in \( G \). \( G' \) is not connected. We have the case that \( G \) is connected but the induced graph \( G' \) is not connected. Therefore the statement is false.
	\item This statement is correct. A forest is a graph without cycles, essentially a collection of trees. An induced subgraph of a forest will also be cycle-free, as removing vertices and their associated edges cannot create a cycle where there was none before. Therefore, any induced subgraph of a forest is also a forest.
	\item If a graph is a tree, then it must also be connected. Because the statement in a) is false, this statement must also be false, i.e. it is not guaranteed that an induced graph of a connected graph is connected. But this would be required for the statement: "If G is a tree, then all its induced subgraphs are trees. "
\end{enumerate}

\section*{Exercise \homeworkNumber.2}

Based on the examples given on the slides for identifying graph invariants to disproof isomorphism, we can propose two more:
\begin{itemize}
    \item Acyclicity
    \item Planarity
\end{itemize}

\section*{Exercise \homeworkNumber.3}
\begin{enumerate}
	\item The graph \( G \) has 2 cycles \( \left\{ \left\{ J , N, L \right\}, \left\{ L, I, K \right\} \right\} \) of length 3. The given graph has three cycles \( \left\{ 1,2,4 \right\}, \left\{ 4,7,6 \right\}, \left\{ 6,7,8 \right\}  \) of length 3. The graphs don't have the same properties and are therefore not isomorphic.

	\item \( \sigma = \left\{ H \to 2, J \to 5, N \to 8, L \to 4, O \to 7, K \to 3, I \to 6, M \to 1 \right\}  \)

\end{enumerate}


\section*{Exercise \homeworkNumber.4}
\begin{enumerate}[(a)]
    \item
    We specify $V1 = \{1,2,4,5\}$ and $V2 = \{3,4,5,6\}$. \\
    $\sigma_{v1 \to v2} = \{ 1 \to 4, 2 \to 5, 4 \to 6, 5 \to 3 \}$
    \item 
    $\sigma' = \{ 4 \to 6, 2 \to 5\}$
\end{enumerate}
\section*{Exercise \homeworkNumber.5}
\begin{enumerate}[(a)]
    \item We multiply the possible compbinations of symetries in every connected component to get the total combinations in the whole graph. This is how to calculate permutations. 
    We can assume the graph has at least the amount of symetries that occur by counting the amount of symetries in the connected components.
    \item As it is possible to have symetries that range across connected components and don't occur within the connected components themselves, this is not a tight lower bound.
\end{enumerate}
\end{document}
