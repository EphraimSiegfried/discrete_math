
%% General definitions
\documentclass{article} %% Determines the general format.
\usepackage{a4wide} %% paper size: A4.
\usepackage[utf8]{inputenc} %% This file is written in UTF-8.
%% Some editors on Windows cannot save files in UTF-8.
%% If there is a problem with special characters not showing up
%% correctly, try switching "utf8" to "latin1" (ISO 8859-1).
\usepackage[T1]{fontenc} %% Format of the resulting PDF file.
\usepackage{fancyhdr} %% Package to create a header on each page.
\usepackage{lastpage} %% Used for "Page X of Y" in the header.
%% For this to work, you have to call pdflatex twice.
\usepackage{enumerate} %% Used to change the style of enumerations (see below).

\usepackage{amssymb} %% Definitions for math symbols.
\usepackage{amsmath} %% Definitions for math symbols.
\usepackage{amsthm} % Definiton for Proofs

\usepackage{tikz}  %% Pagacke to create graphics (graphs, automata, etc.)
\usetikzlibrary{automata} %% Tikz library to draw automata
\usetikzlibrary{arrows}   %% Tikz library for nicer arrow heads

\newtheorem{definition}{Definition}
\newtheorem{theorem}{Theorem}
\newtheorem{axiom}{Axiom}

%% Left side of header
\lhead{\course\\\semester\\Exercise \homeworkNumber}
%% Right side of header
\rhead{\authorname\\Page \thepage\ of \pageref{LastPage}}
%% Height of header
\usepackage[headheight=36pt]{geometry}
%% Page style that uses the header
\pagestyle{fancy}

\newcommand{\authorname}{Alex Lutsch\\Ephraim Siegfried }
\newcommand{\semester}{Fall Semester 2023}
\newcommand{\course}{Discrete Mathematics in Computer Science}
\newcommand{\homeworkNumber}{2}

\begin{document}


\section*{Exercise \homeworkNumber.1}
We examine the following theorem: \\[5pt]
$\sum\limits_{i=0}^n i = \frac{n*(n+1)}{2} \quad \forall \quad n\in \mathbb N_0$
\\
\begin{proof}
	Mathematical induction over n:\\[5pt]
	\textbf{Basis $n-1=0$ :}\\
	\begin{flalign*}
		0 & =\frac{0*1}{2} & \\
		0 & = 0            &
	\end{flalign*}
	\textbf{Induction Hypothesis: } \\[5pt]
	$\sum\limits_{i=0}^k i = \frac{k*(k+1)}{2} $ for $k = n - 1$ \\[5pt]
	\textbf{Inductive Step:} $n-1 \rightarrow{} n$ \\[10pt]
	\begin{align*}
		 & \sum\limits_{i=0}^{n-1} i = \sum\limits_{i=0}^n i - n \stackrel{IH}{=} \frac{n*(n+1)}{2} - n \\[5pt]
		 & \iff \quad \frac{n*(n+1)}{2} - \frac{2n}{2}  \\[5pt]
		 & \iff \quad \frac{n*(n+1)-2n}{2}				\\[5pt]
		 & \iff \quad \frac{n*(n+1-2)}{2}				\\[5pt]
		 & \iff \quad \frac{n*(n-1)}{2}					\\[5pt]
		 & \iff \quad \frac{n-1*((n-1)+1)}{2}			\\[5pt]
	\end{align*}
\end{proof}
\newpage

\section*{Exercise \homeworkNumber.2}
We will prove the following statement by structural induction:
\begin{theorem}
	For all binary trees \( B \) it holds that \( \text{edges}(B) = 2 \cdot \text{leaves}(B) - 2 \).
\end{theorem}

\begin{proof}
	\textbf{Base Case:}
	\begin{align*}
		edges(\Box) = 0 = 2 - 2 = 2 * leaves(\Box) - 2 .     \\
		\leadsto \text{Statement is true for the base case } \\
	\end{align*}
	\textbf{Induction Hypothesis:} Assume that for a composite tree \( \left< L, \circ, R \right> \) the statement is true for the subtrees \( L \) and \( R \).
	\\
	\\
	\textbf{Inductive Step:}
	Consider a composite tree \( B = \langle L, \circ, R \rangle \).
	\begin{align*}
		edges(B) & = edges(L) + edges(B) + 2                                          \\
		         & \stackrel{IH}{=} 2 \cdot leaves(L) - 2 + 2 \cdot leaves(R) - 2 + 2 \\
		         & = 2 \cdot (leaves(L) + leaves(R)) -2                               \\
		         & = 2 \cdot leaves(B) - 2
	\end{align*}

\end{proof}
\newpage

\section*{Exercise \homeworkNumber.3}
A set of words 'S' is defined as follows:
\begin{itemize}
	\item
		'baa' is in S.
	\item
		'c' is in S.
	\item
		If x and y are in S, then so is xyx. 
	\item
		If x is in S, then so is 'b'x'b'. 
\end{itemize}
\begin{theorem}
	All words in S have odd length.
\end{theorem}
We prove by structural induction:
\begin{proof}
	\textbf{Basis:}\\
		'baa' and 'c' are in S, their length is 3 and 1, both of these numbers are odd.\\[10pt]
	\textbf{Induction Hypothesis:}\\
		There are words $x,y \in S$ that are of odd length.\\[10pt]
	\textbf{Inductive Step:}
		\begin{description}
			\item[case1:] We apply the transformation xyx to two words x and y.
			Considering only the length of the words we can rewrite this as 2*x + 1*y.
			Using the Hypothesis that x and y are odd words we can reinterpreted this as 2*odd + 1*odd = 3 * odd.
			Since 3 is an odd number we know odd*odd = odd.
			\item[case2:] We apply the transformation 'b' x 'b' to the word x.
			Using the Hypothesis that x is of odd length and 'b' being odd length of 1 then we can rewrite the length as 2*odd + 1* odd = 3*odd.
			Since 3 is an odd number we know odd*odd = odd.
		\end{description}

\end{proof}
\newpage

\section*{Exercise \homeworkNumber.4}
\begin{enumerate}[(a)]
	\item The set builder notation is wrong because it does not specify that x is a natural number and it also does not define n.
	      The correct notation would be \(\left\{ x \mid x \in \mathbb{N}, x < 20, x \mod 2 = 1 \right\}\).
	\item The notation is wrong since x is undefined and the 6 alone is not a set.
	      The correct notation would be \( \left\{ x \mid x \in \mathbb{N}, x \neq 6 \right\} \).
\end{enumerate}
\newpage

\section*{Exercise \homeworkNumber.5}
\begin{enumerate}[(a)]
	\item We can first find out what the union of A and B is: \( A \cup B = U \setminus (A \cup B)^{c} = \{1, 3, 5, 6, 8, 9, 10\} \).
	      With the latter and \( A \cap B = \{1, 3\}  \) it follows that \( \{1, 3\} \subset A \subset A \cup B \)
	      and \( \{1, 3\} \subset B \subset A \cup B \).
	      The following sets satisfy the properties: \( A = \{1, 3, 5, 6, 8, 9\}, B = \{1, 3, 10\}\).
	\item It is not possible to satisfy all conditions at the same time.
	      If \( A \cap B = \emptyset \), it means that \( A, B \) have no elements in common.
	      However, if \( A \subset B \), it means that every element of \( A \) is an element of \( B \),
	      which contradicts the first condition of \( A \) and \( B \) having no common elements.
	      Therefore, \( A,B \) under the given conditions do not exist.
	\item The following set \( A, B \) satisfy the required properties:
	      \( A = B = \{6,7,8\} \).
\end{enumerate}


\end{document}
