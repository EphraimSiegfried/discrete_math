
%% General definitions
\documentclass{article} %% Determines the general format.
\usepackage{a4wide} %% paper size: A4.
\usepackage[utf8]{inputenc} %% This file is written in UTF-8.
%% Some editors on Windows cannot save files in UTF-8.
%% If there is a problem with special characters not showing up
%% correctly, try switching "utf8" to "latin1" (ISO 8859-1).
\usepackage[T1]{fontenc} %% Format of the resulting PDF file.
\usepackage{fancyhdr} %% Package to create a header on each page.
\usepackage{lastpage} %% Used for "Page X of Y" in the header.
%% For this to work, you have to call pdflatex twice.
\usepackage{enumerate} %% Used to change the style of enumerations (see below).

\usepackage{amssymb} %% Definitions for math symbols.
\usepackage{amsmath} %% Definitions for math symbols.
\usepackage{amsthm} % Definiton for Proofs

\usepackage{tikz}  %% Pagacke to create graphics (graphs, automata, etc.)
\usetikzlibrary{automata} %% Tikz library to draw automata
\usetikzlibrary{arrows}   %% Tikz library for nicer arrow heads


%% Left side of header
\lhead{\course\\\semester\\Exercise \homeworkNumber}
%% Right side of header
\rhead{\authorname\\Page \thepage\ of \pageref{LastPage}}
%% Height of header
\usepackage[headheight=36pt]{geometry}
%% Page style that uses the header
\pagestyle{fancy}

\newcommand{\authorname}{Alex Lutsch\\Ephraim Siegfried }
\newcommand{\semester}{Fall Semester 2023}
\newcommand{\course}{Discrete Mathematics in Computer Science}
\newcommand{\homeworkNumber}{2}


\begin{document}


\section*{Exercise \homeworkNumber.1}
We examine the following theorem: \\[5pt]
$\sum\limits_{i=0}^n i = \frac{n*(n+1)}{2} \quad \forall \quad n\in \mathbb N_0$
\\
\begin{proof}
	Mathematical induction over n:\\[5pt]
	\textbf{Basis $n=0$:}\\
	\begin{flalign*} 
		0&=\frac{0*1}{2} &\\
		0 &= 0 &
	\end{flalign*}
	\textbf{Induction Hypothesis: } \\[5pt]
	$\sum\limits_{i=0}^k i = \frac{k*(k+1)}{2} $ for $k = n - 1$ \\[5pt]
	\textbf{Inductive Step:} $n-1 \rightarrow{} n$ \\[10pt]
	\begin{alignat*}{3}
		&& \sum\limits_{k=1}^n k^3 &= \frac{n^2}{4}(n+1)^2 \quad && \vert +(n+1)^3 \\
		\iff \quad && \sum\limits_{k=1}^{n+1} k^3 &= \frac{n^2}{4}(n+1)^2 + (n+1)^3 \quad && \\
		\iff \quad && \sum\limits_{k=1}^{n+1} k^3 &= \frac{n^2}{4}(n^2+2n+1) + (n+1)^3 \quad && \\
	\end{alignat*}
\end{proof}

\section*{Exercise \homeworkNumber.2}


\section*{Exercise \homeworkNumber.3}


\section*{Exercise \homeworkNumber.4}
\begin{enumerate}[(a)]
	\item The set builder notation is wrong because it does not specify that x is a natural number and it also does not define n.
	      The correct notation would be \(\left\{ x \mid x \in \mathbb{N}, x < 20, x \mod 2 = 1 \right\}\).
	\item The notation is wrong since x is undefined and the 6 alone is not a set.
	      The correct notation would be \( \left\{ x \mid x \in \mathbb{N}, x \neq 6 \right\} \).
\end{enumerate}

\section*{Exercise \homeworkNumber.5}
\begin{enumerate}[(a)]
	\item We can first find out what the union of A and B is: \( A \cup B = U \setminus (A \cup B)^{c} = \{1, 3, 5, 6, 8, 9, 10\} \).
	      With the latter and \( A \cap B = \{1, 3\}  \) it follows that \( \{1, 3\} \subset A \subset A \cup B \)
	      and \( \{1, 3\} \subset B \subset A \cup B \).
	      The following sets satisfy the properties: \( A = \{1, 3, 5, 6, 8, 9\}, B = \{1, 3, 10\}\).
	\item It is not possible to satisfy all conditions at the same time.
	      If \( A \cap B = \emptyset \), it means that \( A, B \) have no elements in common.
	      However, if \( A \subset B \), it means that every element of \( A \) is an element of \( B \),
	      which contradicts the first condition of \( A \) and \( B \) having no common elements.
	      Therefore, \( A,B \) under the given conditions do not exist.
	\item The following set \( A, B \) satisfy the required properties:
	      \( A = B = \{6,7,8\} \).
\end{enumerate}


\end{document}
