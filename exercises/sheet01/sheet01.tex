
%% General definitions
\documentclass{article} %% Determines the general format.
\usepackage{a4wide} %% paper size: A4.
\usepackage[utf8]{inputenc} %% This file is written in UTF-8.
%% Some editors on Windows cannot save files in UTF-8.
%% If there is a problem with special characters not showing up
%% correctly, try switching "utf8" to "latin1" (ISO 8859-1).
\usepackage[T1]{fontenc} %% Format of the resulting PDF file.
\usepackage{fancyhdr} %% Package to create a header on each page.
\usepackage{lastpage} %% Used for "Page X of Y" in the header.
%% For this to work, you have to call pdflatex twice.
\usepackage{enumerate} %% Used to change the style of enumerations (see below).

\usepackage{amssymb} %% Definitions for math symbols.
\usepackage{amsmath} %% Definitions for math symbols.
\usepackage{amsthm} % Definiton for Proofs

\usepackage{tikz}  %% Pagacke to create graphics (graphs, automata, etc.)
\usetikzlibrary{automata} %% Tikz library to draw automata
\usetikzlibrary{arrows}   %% Tikz library for nicer arrow heads


%% Left side of header
\lhead{\course\\\semester\\Exercise \homeworkNumber}
%% Right side of header
\rhead{\authorname\\Page \thepage\ of \pageref{LastPage}}
%% Height of header
\usepackage[headheight=36pt]{geometry}
%% Page style that uses the header
\pagestyle{fancy}

\newcommand{\authorname}{Alex Lutsch\\Kemal Yönet\\Ephraim Siegfried }
\newcommand{\semester}{Fall Semester 2023}
\newcommand{\course}{Discrete Mathematics in Computer Science}
\newcommand{\homeworkNumber}{1}


\begin{document}



\section*{Exercise \homeworkNumber.1}
We will proof the following statement with a direct proof:
\begin{center}
	For all sets \( A, B, C \) it holds that \( (A \cap B) \cup (A \cap C) \subset A \cap (B \cup C) \)
\end{center}

\begin{proof}
	Let \( A, B, C \) be arbitrary sets. We will show that \( x \in  (A \cap B) \cup (A \cap C)\) implies \( x \in A \cap (B \cup C)\).
	\\
	\\
	Consider any \( x \in (A \cap B) \cup (A \cap C) \).
	By the definition of the union it holds that \( x \in (A \cap B) \) or \( x \in (A \cap C) \).
	With the definition of the intersection this means that \( (x \in A \text{ and } x \in B) \text{ or } (x \in A \text{ and } x \in  C) \).
	In both cases of the disjunction we have that \( x \in A \),
	therefore this implies that \( x \in  A \text{ and } (x \in A \text{ or }x \in C) \),
	which is equivalent to \( x \in A \cap (A \cup B) \).
\end{proof}

\section*{Exercise \homeworkNumber.2}
Proof by contradiction to establish the following theorem:
\begin{center}
	For all sets $A$ and $B$: If $(A \cap B) = \emptyset$, then $(A \setminus B) = A$.
\end{center}
\begin{proof}
	Assume if $A \cap B = \emptyset$, then $A\setminus B \neq A$ \\
	Since $(A \setminus B) \neq A$ there is a $x \in B$ that's also $x \in A$. But since $A \cap B = \emptyset$ this is impossible. Hence the original statement must hold.
\end{proof}



\section*{Exercise \homeworkNumber.3}
We will proof the following statement by contrapositive:
\begin{center}
	For all sets \( A,B \) we have if \( A \cup B = B \) then \( A \subseteq B \)
\end{center}
\begin{proof}
	Let \( A,B \) be arbitrary sets.
	We will show that if \( A \nsubseteq B  \) then \( A \cup B \neq B \).
	Since \( A \nsubseteq B \) there exists at least one \( x \) with \( x \in A \) and \( x \not\in B \).
	Note that \( A \subseteq A \cup B \) since all elements of \( A \) are elements of \( A \) or \( B \).
	This means that \( A \cup B \) contains at least one element which is not in \( B \).
	Therefore not all elements of \( A \cup B \) are elements of B, thus \( A \cup B \nsubseteq B \)
	and by definition \( A \cup B \neq B \).

\end{proof}



\section*{Exercise \homeworkNumber.4}
We examine the following statement:
\begin{center}
	For all sets A, B and C: if $A \subseteq (B \cup C)$, then $A \subseteq B$ or $A \subseteq C$.
\end{center}
We choose a set $A = B \cup C$ where $B \neq C$ and $B,C \neq \emptyset$. It follows that $A \not\subseteq B$ and $A \not\subseteq C$ since B and C are now strict subsets of A. Since $A \subseteq (B \cap C)$ holds this contradicts the statement.




\end{document}
