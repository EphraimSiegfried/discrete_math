
%% General definitions
\documentclass{article} %% Determines the general format.
\usepackage{a4wide} %% paper size: A4.
\usepackage[utf8]{inputenc} %% This file is written in UTF-8.
%% Some editors on Windows cannot save files in UTF-8.
%% If there is a problem with special characters not showing up
%% correctly, try switching "utf8" to "latin1" (ISO 8859-1).
\usepackage[T1]{fontenc} %% Format of the resulting PDF file.
\usepackage{fancyhdr} %% Package to create a header on each page.
\usepackage{lastpage} %% Used for "Page X of Y" in the header.
%% For this to work, you have to call pdflatex twice.
\usepackage{enumerate} %% Used to change the style of enumerations (see below).

\usepackage{amssymb} %% Definitions for math symbols.
\usepackage{amsmath} %% Definitions for math symbols.

\usepackage{tikz}  %% Pagacke to create graphics (graphs, automata, etc.)
\usetikzlibrary{automata} %% Tikz library to draw automata
\usetikzlibrary{arrows}   %% Tikz library for nicer arrow heads


%% Left side of header
\lhead{\course\\\semester\\Exercise \homeworkNumber}
%% Right side of header
\rhead{\authorname\\Page \thepage\ of \pageref{LastPage}}
%% Height of header
\usepackage[headheight=36pt]{geometry}
%% Page style that uses the header
\pagestyle{fancy}

\newcommand{\authorname}{Alex Lutsch\\Ephraim Siegfried }
\newcommand{\semester}{Fall Semester 2023}
\newcommand{\course}{Discrete Mathematics in Computer Science}
\newcommand{\homeworkNumber}{11}


\begin{document}



\section*{Exercise \homeworkNumber.1}
\begin{align*}
	(\neg((\neg B \land D) \land (C \lor D)) \land (B \lor A)) & \equiv (\neg(\neg B \land (D \land (C \lor D)) \land (B \lor A))) & \text{(Associativity)}    \\
	                                                           & \equiv (\neg(\neg B \land (D \land (D \lor C)) \land (B \lor A))) & \text{(Commutativity) }   \\
	                                                           & \equiv (\neg(\neg B \land D) \land (B \lor A)))                   & \text{(Absorption) }      \\
	                                                           & \equiv ((\neg\neg B \lor \neg D) \land (B \lor A))                & \text{ (De Morgan) }      \\
	                                                           & \equiv ((B \lor \neg D) \land (B \lor A))                         & \text{(Double Negation) } \\
	                                                           & \equiv (B \lor ( \neg D \land A))                                 & \text{(Distributivity) }  \\
\end{align*}


\section*{Exercise \homeworkNumber.2}
I will disprove the statement: Every single Literal is a clause and a monomial, because the terms clause and monomial are also used for the corner case with only one literal. Every literal is also a formula. Therefore there exists a formula which is both a monomial and a clause.




\section*{Exercise \homeworkNumber.3}
\begin{enumerate}[(a)]
	\item
	\item
\end{enumerate}



\section*{Exercise \homeworkNumber.4}



\section*{Exercise \homeworkNumber.5}
\begin{enumerate}[(a)]
	\item Because of distributivity we can transform the formula from CNF to DNF like this: \( \phi_n = \bigwedge\limits_{i=1}^{n}(a_{0} \lor a_i) \equiv \bigvee\limits_{S \in \mathcal{P}(\left\{ 1, \ldots ,n \right\} )}(a_{0} \land \bigwedge\limits_{i \in \left\{ 1, \ldots ,n \right\} } a_{i})\). Because the cardinality of the power set is \( \mathcal{P}({1,\ldots ,n}) = 2^n \), we will have \( 2^n \) monomials.
	\item The equivalent form is \( \phi_n = \bigwedge\limits_{i=1}^{n}(a_{0} \lor a_i) \equiv a_{0} \lor \bigwedge\limits_{i=1}^{n}a_{i}\) and has has size polynomial in n. The equivalence can be explained by the fact that if \( a_{0} \) is true, the original formula is satisfied (since \( a_{0} \) appears in every clause of the CNF), and if \( a_{0} \) is false, then all \( a_i \) must be true for the original formula to be satisfied, which is exactly what the specified formula expresses.

\end{enumerate}


\end{document}
