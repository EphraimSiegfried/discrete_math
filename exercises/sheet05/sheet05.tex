
%% General definitions
\documentclass{article} %% Determines the general format.
\usepackage{a4wide} %% paper size: A4.
\usepackage[utf8]{inputenc} %% This file is written in UTF-8.
%% Some editors on Windows cannot save files in UTF-8.
%% If there is a problem with special characters not showing up
%% correctly, try switching "utf8" to "latin1" (ISO 8859-1).
\usepackage[T1]{fontenc} %% Format of the resulting PDF file.
\usepackage{fancyhdr} %% Package to create a header on each page.
\usepackage{lastpage} %% Used for "Page X of Y" in the header.
%% For this to work, you have to call pdflatex twice.
\usepackage{enumerate} %% Used to change the style of enumerations (see below).

\usepackage{amssymb} %% Definitions for math symbols.
\usepackage{amsmath} %% Definitions for math symbols.

\usepackage{tikz}  %% Pagacke to create graphics (graphs, automata, etc.)
\usetikzlibrary{automata} %% Tikz library to draw automata
\usetikzlibrary{arrows}   %% Tikz library for nicer arrow heads


%% Left side of header
\lhead{\course\\\semester\\Exercise \homeworkNumber}
%% Right side of header
\rhead{\authorname\\Page \thepage\ of \pageref{LastPage}}
%% Height of header
\usepackage[headheight=36pt]{geometry}
%% Page style that uses the header
\pagestyle{fancy}

\newcommand{\authorname}{Alex Lutsch\\Ephraim Siegfried }
\newcommand{\semester}{Fall Semester 2023}
\newcommand{\course}{Discrete Mathematics in Computer Science}
\newcommand{\homeworkNumber}{5}


\begin{document}



\section*{Exercise \homeworkNumber.1}

\begin{enumerate}[(a)]
	\item
	From the slides we have the hint that equivalence relations induce partitions.\\
	We look for the amount of relations over the equivalence set of M: $\lvert R \lvert \subseteq (\sim M), M =\{1,2,3\}$.
	Since $(\sim M) = M_1 \times M_2$, where $M_1 \times M_2$ has to be reflexive, symetric and transitive, every possible subset of $(\sim M)$ has to fulfill $M_1 = M_2$. \\
	This is equivalent to the different ways you can partition the set M into disjoint subsets. \\
	$R_1 = \{\{1,2\},3\}, R_2 = \{1,\{2,3\}\}$ etc.., $\lvert R \lvert = 5$.
	\item
	$[a]_\sim = \{a\}$,
	$[b]_\sim = \{b,d\}$,
	$[c]_\sim = \{c,e\}$,
	$[d]_\sim = \{d,b\}$,
	$[e]_\sim = \{e,c\}$


\end{enumerate}



\section*{Exercise \homeworkNumber.2}

\begin{enumerate}[(a)]
	\item Since the equivalence relation \( \sim \) is reflexive, we have that for all \( x \in S \) that \( (x, x) \in \sim \).
	      Therefore, for any equivalence class \( [x]_{\sim} =  \left\{ y \in S \mid x \sim y \right\}  \) there must exist a \( y \in [x]_{\sim}\) with \( x = y \).
	      Each equivalence class of the set \( E = \left\{  [x]_{\sim} \mid x \in S \right\} \) has the latter property.
	      Therefore, every element of \( S \) is in some equivalence class in \( E \).
	\item I will show this by an indirect proof.
	      Assume there is are \( x, y, z \in S\) with \( x \in [y]_{\sim}, x \in [z]_{\sim} \) and \( [y]_{\sim} \neq [z]_{\sim} \).
	      Because equivalence relations are symmetric, we have that for arbitrary \( s,q \) that \( s \in [q]_{\sim} \) iff \( q \in [s]_{\sim} \).
	      Therefore \( x \in [y]_{\sim} \leftrightarrow y \in [x]_{\sim}\) and \( x \in [z]_{\sim} \leftrightarrow z \in [x]_{\sim}\).
	      We have a contradicition, since we assumed \( [y]_{\sim} \neq [z]_{\sim} \) which is equivalent to \( [x]_{\sim} \neq [x]_{\sim} \), but it is the case that \( [x]_{\sim} = [x]_{\sim} \) .
	      Thus every element of \( S \) is in at most one equivalence class in \( E \).
\end{enumerate}



\section*{Exercise \homeworkNumber.3}
$S = \{(c,c),(d,d),(a,b)\}$ \\
It is a partial order since not every element is related to another and its reflexive, antisymetric and transitive. \\
c and d are the minmial elements as there is no $y$ with $ y \preceq c,d  $.


\section*{Exercise \homeworkNumber.4}

\begin{enumerate}[(a)]
	\item Let \( R \) be a total order over an arbitrary set \( S \) and let \( x,y \in S \).
	      If \( x = y \), then the statements \( xRy  \) and \( yRx \) are the same statements and thus count as only one statement.
	      If \( x \neq y \), then we have, because partial relations are antisymmetric, that if \( xRy \in R\), then \( yRx \not\in R\) and vice versa.
	      Therefore, either \( xRy \in R \) xor \( yRx \in R \) is true.
	\item I will disprove this statement by a counterexample.
	      Let \( S = \left\{ a,b \right\}  \).
	      The only strict orders over S which exist are \( R_{1} = \left\{ (a,b) \right\}  \), \( R_{2} = \left\{ (b, a) \right\}  \). \\
	      Let's look at the strict order \( R_{1} \).
	      Here \( a \) is minimal, since there is no \( y \in R_{1} \) with \( yR_{1}a \).
	      The element \( a \) is not maximal, since there exists \( y \in R_{1} \) with \( aR_{1}y \) which is \( y = b \). \\
	      An analogous argument can be made for \( R_{2} \).
	      Therefore, there does not exist a strict order over S where all \( x \in S \) are both minimal and maximal. Thus the statement is false.

\end{enumerate}


\section*{Exercise \homeworkNumber.5}

\begin{enumerate}[(a)]
	\item
	$ A^{-1} = \{\langle 2x,x \rangle \lvert x \in \mathbb N_o \}$
	\item
	$ B \setminus A^{-1} = \{\langle i*x,x \rangle \lvert i,x \in \mathbb N_0, i \neq 2\}$
	\item
	$C \circ A = \{\langle 4,4 \rangle,\langle 5,16 \rangle, \langle 4,14\rangle, \langle 8,4 \rangle,\langle 6,8\rangle\}$
	\item
	$A \circ (A \circ A) = \{ \langle x,8*x\rangle\lvert x\in \mathbb N_0\}$
 	\item
	 $A^{*} = \{\langle x ,2^y*x\rangle\lvert x,y \in \mathbb N_0\}$
	 \item 
	 $A \circ B = \{\langle x,x\rangle \lvert (2*x = i*x) : i,x \in \mathbb N_0\}$

\end{enumerate}

\end{document}
