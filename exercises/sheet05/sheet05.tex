
%% General definitions
\documentclass{article} %% Determines the general format.
\usepackage{a4wide} %% paper size: A4.
\usepackage[utf8]{inputenc} %% This file is written in UTF-8.
%% Some editors on Windows cannot save files in UTF-8.
%% If there is a problem with special characters not showing up
%% correctly, try switching "utf8" to "latin1" (ISO 8859-1).
\usepackage[T1]{fontenc} %% Format of the resulting PDF file.
\usepackage{fancyhdr} %% Package to create a header on each page.
\usepackage{lastpage} %% Used for "Page X of Y" in the header.
%% For this to work, you have to call pdflatex twice.
\usepackage{enumerate} %% Used to change the style of enumerations (see below).

\usepackage{amssymb} %% Definitions for math symbols.
\usepackage{amsmath} %% Definitions for math symbols.

\usepackage{tikz}  %% Pagacke to create graphics (graphs, automata, etc.)
\usetikzlibrary{automata} %% Tikz library to draw automata
\usetikzlibrary{arrows}   %% Tikz library for nicer arrow heads


%% Left side of header
\lhead{\course\\\semester\\Exercise \homeworkNumber}
%% Right side of header
\rhead{\authorname\\Page \thepage\ of \pageref{LastPage}}
%% Height of header
\usepackage[headheight=36pt]{geometry}
%% Page style that uses the header
\pagestyle{fancy}

\newcommand{\authorname}{Alex Lutsch\\Ephraim Siegfried }
\newcommand{\semester}{Fall Semester 2023}
\newcommand{\course}{Discrete Mathematics in Computer Science}
\newcommand{\homeworkNumber}{5}


\begin{document}



\section*{Exercise \homeworkNumber.1}

\begin{enumerate}[(a)]
	\item
	\item
\end{enumerate}



\section*{Exercise \homeworkNumber.2}

\begin{enumerate}[(a)]
	\item
	\item
\end{enumerate}



\section*{Exercise \homeworkNumber.3}



\section*{Exercise \homeworkNumber.4}

\begin{enumerate}[(a)]
	\item Let \( R \) be a total order over an arbitrary set \( S \) and let \( x,y \in S \).
	      If \( x = y \), then the statements \( xRy  \) and \( yRx \) are the same statements and thus count as only one statement.
	      If \( x \neq y \), then we have, because partial relations are antisymmetric, that if \( xRy \in R\), then \( yRx \not\in R\) and vice versa.
	      Therefore, either \( xRy \in R \) xor \( yRx \in R \) is true.
	\item I will disprove this statement by a counterexample.
	      Let \( S = \left\{ a,b \right\}  \).
	      The only strict orders over S which exist are \( R_{1} = \left\{ (a,b) \right\}  \), \( R_{2} = \left\{ (b, a) \right\}  \). \\
	      Let's look at the strict order \( R_{1} \).
	      Here \( a \) is minimal, since there is no \( y \in R_{1} \) with \( yR_{1}a \).
	      The element \( a \) is not maximal, since there exists \( y \in R_{1} \) with \( aR_{1}y \) which is \( y = b \). \\
	      An analogous argument can be made for \( R_{2} \).
	      Therefore, there does not exist a strict order over S where all \( x \in S \) are both minimal and maximal. Thus the statement is false.

\end{enumerate}


\section*{Exercise \homeworkNumber.5}

\begin{enumerate}[(a)]
	\item
	\item
	\item
	\item
	\item
	\item
\end{enumerate}

\end{document}
