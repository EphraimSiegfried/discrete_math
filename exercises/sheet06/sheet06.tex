
%% General definitions
\documentclass{article} %% Determines the general format.
\usepackage{a4wide} %% paper size: A4.
\usepackage[utf8]{inputenc} %% This file is written in UTF-8.
%% Some editors on Windows cannot save files in UTF-8.
%% If there is a problem with special characters not showing up
%% correctly, try switching "utf8" to "latin1" (ISO 8859-1).
\usepackage[T1]{fontenc} %% Format of the resulting PDF file.
\usepackage{fancyhdr} %% Package to create a header on each page.
\usepackage{lastpage} %% Used for "Page X of Y" in the header.
%% For this to work, you have to call pdflatex twice.
\usepackage{enumerate} %% Used to change the style of enumerations (see below).

\usepackage{amssymb} %% Definitions for math symbols.
\usepackage{amsmath} %% Definitions for math symbols.

\usepackage{tikz}  %% Pagacke to create graphics (graphs, automata, etc.)
\usetikzlibrary{automata} %% Tikz library to draw automata
\usetikzlibrary{arrows}   %% Tikz library for nicer arrow heads


%% Left side of header
\lhead{\course\\\semester\\Exercise \homeworkNumber}
%% Right side of header
\rhead{\authorname\\Page \thepage\ of \pageref{LastPage}}
%% Height of header
\usepackage[headheight=36pt]{geometry}
%% Page style that uses the header
\pagestyle{fancy}

\newcommand{\authorname}{Alex Lutsch\\Ephraim Siegfried }
\newcommand{\semester}{Fall Semester 2023}
\newcommand{\course}{Discrete Mathematics in Computer Science}
\newcommand{\homeworkNumber}{6}


\begin{document}



\section*{Exercise \homeworkNumber.1}
dom(f) $= \mathbb N_1$ \\
img(f) $= \{3x \lvert x \in \mathbb N_0\} $



\section*{Exercise \homeworkNumber.2}
$ f(0) = v, f(1) = x, f(2) = u, f(4) = z, f(5) = z $ \\
The lenght of the image is 4, 3 isnt in the domain, the inverse function has preimage of 3 and image of 4 so we have a double mapping, this double mapping doesnt appear in $f\lvert\{0,1,2\}$ as it has to be a total function.




\section*{Exercise \homeworkNumber.3}
Let \( f \colon \mathbb{R}_{\geq 0} \to \mathbb{R}_{\geq 0} \colon f(x) = \sqrt{x}  \)
and \( g \colon \mathbb{R} \to \mathbb{R} \colon g(x) = x^2  \).
We have that g(x) is not injective as it was shown in the lecture.
The composition \( g \circ f = g(f(x)) = (\sqrt{x})^2 = x \) is injective,
since we have that for all \( x,y \in  \mathbb{R}_{\geq 0} \) with \( x \neq y \) that \( f(x) = x \neq y = f(y) \).



\section*{Exercise \homeworkNumber.4}
We have a function g which maps every f(x) to x, for all x which are in the domain of f. This means every x has to be assigned to at most one f(x), otherwise we would have ambiguity. Hence the function f has to be injective.\\
We also have a function h with which we can map every value $y$ in the codomain of f to itself ($y$) by inserting h into f: $f(h(y)) = y$. This means for every $y \in$ "codomain of f" there is an $f(h(y)) = y$, which implies that every possible value in the codomain of f must have a mapping. We conclude that the function f is also surjective. \\
As the function f is injective and surjective, it is bijective.


\section*{Exercise \homeworkNumber.5}
I solved (b) and (c) with a python script which is in the appendix.
\begin{enumerate}[(a)]
	\item Since A and D, B and C are switched, we have that \( C = \left\{ A \to D, B \to C, C \to B, D \to A, E \to E \right\}  \). Therefore \( P \circ C = \left\{ A \to 4, B \to 3, C \to 2, D \to 1, E \to 5 \right\} \).
	\item \( R_{1} = \left\{ 1 \to 1, 2 \to 3, 3 \to 4, 4 \to 5, 5 \to 2 \right\}  \)
	\item The encoding of ED is EC.
\end{enumerate}

\end{document}
